State-of-the-art silicon monolithic electronic-silicon photonics processes are a monumental advancement in the co-design of electronic and photonic circuits.
For high speed applications, sharing the same substrate has its advantages: no interconnection needed reducing the parasitic load, integrated simulation environment and simpler mechanical assembly of the integrated circuit (IC) into more complex systems.
Although at the moment, even the best monolithic processes struggle to compete with recent CMOS nodes in terms of maximum frequency of operation for digital signal processing or the driving power needed for optical modulators. 
For lower speed applications that do not need these higher speeds or driving power, monolithic nodes seem to be an enticing solution for its ease of use, embedded testing capabilities and easier physical assembly.
Nevertheless, since monolithic nodes are usually built upon an existing CMOS node and modified to fit the less strenuous fabrication of the photonics, they tend to exceed the price per area of similar CMOS processes\cite{shekhar_roadmapping_2024}. 

Integrating CMOS capabilities in a zero-change silicon photonics node, although possible, is relatively new\cite{zanetto_unconventional_2023,shekhar_roadmapping_2024}. 
It grants researchers and industry access to a monolithic node without the high entry price at the cost of poorer electrical performances than equivalent CMOS and has no foundry supplied electronic libraries.
Nevertheless, it has recently been used in power monitoring and signal multiplexing applications\cite{crico_monolithic_2024,zanetto_timemultiplexed_2023}.

In this paper, we aim to reduce the gap between commercial monolithic nodes and a enhanced silicon photonics node by showcasing a digital core, proposing improvements on the transistor design and discussing the limitations of the current implementation and tooling.



%It forces the designers of integrated circuit systems to either choose multiple technologies, one dedicated for photonics and others for the CMOS DSP and slow control, and have the challenge of doing the co-design and interconnect or bend to monolithic . 
%As such, a system designer would want to use their monolithic area as best as they can, but they are confronted by three 
%The larger size of the silicon photonics circuits compared to their CMOS counterpart means that one pays the monolithic premium price 


%As the process is geared toward both CMOS and optical capabilities, the price of the area is superior to thoses of either the similar size CMOS or similar performance silicon photonics. 
%This creates a difficult choice for designers who wishes to make an integrated circuit system, either choose two technologies, one for the photonics and the other for the CMOS, and have the hassle of doing the co-design and interconnect or choosing the monolithic process and pay the increased price.
